\chapter{PENDAHULUAN}

\section{Latar Belakang}
Matematika sebagai salah satu pelajaran pokok pada satuan pendidikan memegang peranan yang sangat penting dalam kelangsungan pendidikan siswa, karena matematika merupakan metode berfikir logis, kritis, kreatif, keteraturan, seni, dan bahasa yang tidak hanya membantu penelitian di bidang ilmu dan teknologi tetapi juga untuk pembentukan keuletan, kepribadian dan karakter siswa. Dalam konteks ini maka setiap jenjang pendidikan, matematika menjadi salah satu mata pelajaran pokok yang wajib diikuti dan dipelajari oleh setiap siswa sekolah dasar \cite{Kadir2011}.

Mengingat akan manfaat matematika tersebut, maka siswa pada tingkat pendidikan dasar dan menengah dituntut untuk menguasai matematika dengan baik. Untuk itu, diperlukan usaha tertentu untuk mempelajari dan menguasai matematika dalam segala bentuk kegiatan belajar. Dalam hal ini peranan guru sangatlah penting terutama dalam proses pembelajaran. Guru sebagai tenaga pengajar yang secara langsung melaksanakan proses pendidikan, maka guru harus dapat memotivasi siswa untuk berpartisipasi aktif dalam proses pembelajaran. Untuk menanamkan pemahaman akan konsep matematika diperlukan suatu pendekatan pembelajaran yang tepat dalam menyampaikannya kepada siswa. Dalam proses pembelajaran penggunaan pendekatan yang tepat merupakan faktor yang utama dan sangat berpengaruh terhadap peningkatan hasil belajar siswa. Proses pembelajaran matematika yang bermakna hanya akan terjadi jika proses belajar matematika di kelas berhasil membelajarkan siswa, baik dalam berpikir maupun dalam bersikap. Karena belajar bukan hanya menyerap informasi secara pasif, melainkan aktif menciptakan pengetahuan dan keterampilan. Salah satu alternatif belajar yang dapat digunakan oleh guru untuk mengatasi kepasifan siswa pasif adalah dengan menggunakan pendekatan problem posing yang merupakan perumusan masalah matematika oleh siswa dari situasi yang tersedia. Menurut asosiasi guru-guru matematika di Amerika Serikat, yaitu \textit{National Council of Teachears of Mathematics} (NCTM), \textit{problem posing} (membuat soal cerita matematika) merupakan “\textit{The Heart of Doing Mathematics}”, yaitu inti dari matematika. Oleh karena itu, NCTM merekomendasikan agar para siswa diberi kesempatan yang sebesar-besarnya untuk mengalami membuat soal sendiri. Dengan pengajaran problem posing ini dapat memberi rangsangan belajar yang lebih terarah bagi siswa dalam meningkatkan hasil belajar matematikanya. Untuk mempelajari secara empiris apakah pengajaran dengan menggunakan pendekatan problem posing dapat efektif meningkatkan hasil belajar matematika siswa, diadakan suatu penelitian mengenai penggunaan pendekatan \textit{problem posing} dalam pembelajaran matematika\cite{Kadir2011}.

Pendidikan merupakan proses yang diperlukan individu untuk mengembangkan potensi diri dan memiliki tujuan tertentu yang diarahkan untuk mendapatkan kesempurnaan dan keseimbangan dalam individu maupun masyarakat \cite{Nurkholis2013}. Dalam pendidikan, siswa diajarkan bagaimana menyelesaikan suatu masalah. Masalah dipahami dengan memahami akan apa yang ditanyakan dan apa yang sudah diketahui. Sedangkan perencanaan dari penyelesaian dari suatu masalah ditunjukkan dengan mengorganisirkan informasi dan data yang ada menggunakan strategi-strategi untuk menemukan kemungkinan penyelesaian \cite{Siswono2005}. Oleh karena itu dalam menyelesaikan suatu masalah, setiap siswa memiliki cara dan proses berpikir yang berbeda, Sehingga perlu untuk mengetahui proses berpikir masing-masing siswa. Untuk mengetahui cara berpikir siswa dapat dilakukan dengan pengamatan aktivitas mereka dalam belajar.

Salah satu aktivitas belajar siswa sekolah dasar adalah membuat soal latihan cerita matematika melalui media pembelajaran. Media pembelajaran yang digunakan dalam penelitian adalah Monsakun. Monsakun merupakan sebuah media pembelajaran digital interaktif yang menerapkan permasalahan aritmatika melalui integrasi kalimat-kalimat soal cerita matematika sederhana \cite{SupiantoHayashiHirashima2016}. Dari haspembelajaran dengan Monsakun tersebut menghasilkan dataset berupa Log data yang berisikan setiap langkah proses berpikir siswa dalam menyelesaikan tugas-tugas selama belajar interaktif dengan Monsakun tersebut. Dalam proses penggunaan Monsakun, setiap siswa mendapatkan soal yang sama. Strategi-strategi yang dilakukan untuk mendapatkan kemungkinan penyelesaian soal menjadi pembeda dari masing-masing siswa. Perbedaan strategi yang dilakukan siswa untuk menyelesaikan soal dalam Monsakun timbul dari cara dan proses berpikir siswa yang berbeda. \textit{Clustering} kemiripan cara berpikir siswa dapat dilakukan berdasarkan data jejak aktivitas yang dilakukan siswa dalam menyelesaikan soal tersebut.
Cara yang dapat dilakukan untuk melakukan mengelompokkan kemiripan cara berpikir siswa adalah dengan menggunakan teknik \textit{Clustering}, melakukan \textit{Clustering} yang digunakan seharusnya dapat memetakan secara visual struktur kompleks dari data. Hal ini dikarenakan pemetaan visual merupakan salah satu cara analisis yang paling efisien untuk membantu menemukan pola maupun informasi dari data \cite{Kreuseler2002}. Agar dapat melakukan visualisasi dari data berdimensi tinggi diperlukan algoritme untuk melakukan pengurangan dimensi data menjadi data berdimensi rendah, dimensi tujuan visualisasi yang umum digunakan adalah 2 dimensi dan 3 dimensi.

Terdapat banyak algoritme yang dapat digunakan untuk melakukan pengurangan dimensi data, salah satu algoritme paling awal untuk pengurangan dimensi adalah \textit{Principal Component Analysis} (PCA). PCA adalah algoritme yang melakukan pengurangan dimensi dari data, pengurangan ini dilakukan dengan melakukan identifikasi arah (\textit{Principal Components}), PCA memiliki kekurangan yaitu selalu mencari nilai linear principal components dari data, sedangkan pada beberapa data nonlinear PCA tidak melakukan akses pada informasi yang tertanam pada data. Algoritme pengurangan dimensi konvensional lainnya adalah \textit{Linear Discriminant Analysis} (LDA).

Seperti halnya PCA, LDA juga merupakan transformasi ortogonal dari dimensi tinggi ke dimensi rendah yang dibatasi secara linier. Perbedaannya terletak pada pengaksesan informasi kategorikal dari data, sehingga pada LDA, data poin yang dikategorikan sama akan dipetakan berdekatan \cite{Hartono2017}. Berbeda 
dengan PCA dan LDA yang melakukan representasi data melalui kombinasi linear dari data,\textit{ Self-Organizing Map} (SOM) tidak terikat oleh linearitas dari data (non-linear). SOM merupakan salah satu algoritme yang sering digunakan untuk melakukan pemetaan data. Melalui SOM melakukan pengurangan dimensi dari dimensi tinggi ke dimensi rendah dengan tetap mempertahankan karakteristik topologi data.

Teknik \textit{Clustering} dilakukan untuk membentuk kelompok berdasarkan kemiripan data jejak aktivitas siswa. Salah satu algoritma yang dapat melakukan Clustering adalah \textit{Self Organizing Map} (SOM). Penelitian sebelumnya tentang SOM pernah di lakukan oleh Wiji Lestari membuktikan bahwa Kelompok yang dihasilkan dengan menggunakan metode SOM mampu memetakan kecerdasan majemuk mahasiswa berdasarkan kemiripan kecerdasan majemuknya \cite{Lestari2014}. Pada pengujian penelitian tersebut, peneliti mengungkapkan bahwa hasil penelitiannya memberikan hasil output yang tetap dan mantap akan tetapi tidak menyajikan nilai hasil evaluasi dari pengujian. Fungsi utama SOM untuk mereduksi dimensi data, artinya mirip PCA yang meng- ekstrak informasi yang esensial dari data. SOM satu bentuk dari vector quantization. Reduksi dengan menggunakan SOM struktur topologi di dimensi tinggi yang sukar dianalisis akan direduksi ke ruang dimensi rendah tapi tetap dengan menjaga struktur topologi data. Sehingga didapat keunggulan SOM selesai sampai di sini. Tidak ada cara baku menganalisis struktur yang terjadi selain dengan visualisasi. karena itu perlu metode seperti graf analisis yang salah satunya menggunakan \textit{m-Array Tree} (m-AT). Sehingga diperlukan pengembangan algoritme SOM-m-AT yang akan menghasilkan \textit{Clustering}, visualisasi, sistem penasehat guru dan analisis log data Monsakun yang lebih efektif.

Permasalahan belajar berupa kesulitan siswa sekolah dasar dalam menyusun soal cerita matematika, perlu rekomendasi berdasarkan pengalaman siswa sekolah dasar di Jepang dalam menggunakan media pembelajaran digital interaktif Monsakun Hal ini mendorong penulis mengusulkan penelitian pengembangan algoritma SOM-m-aT dengan menggunakan log data aktivitas siswa dari mengerjakan soal cerita aritmatika dalam media pembelajaran interaktif Monsakun untuk membentuk kelompok siswa dan sistem penasehat pembelajaran guru dalam problem posing melalui integrasi kalimat cerita Aritmatika. \textit{Novelty} dalam penelitian ini adalah implementasi algoritma SOM-m-aT  dalam \textit{Clustering} dan sistem penasehat pembelajaran guru. Pengujian kualitas Kelompok yang terbentuk dengan menggunakan metode \textit{Silhoutte Coeffisient} dan \textit{Davies-Bouldin Index}, sehingga \textit{Clustering} siswa dan nasehat dari guru berdasarkan cara berpikirnya di masa yang akan datang dapat membantu perkembangan di bidang pendidikan karena siswa terkelompok berdasarkan kemiripan cara berpikir atas nasehat guru, sehingga proses pembelajaran akan menjadi lebih efektif. 

Teknik \textit{Clustering} sebagai alat untuk menganalisis perilaku siswa dalam pembelajaran, berpotensi di masa depan. Analisisnya bisa berupa digunakan sebagai dasar bagi fasilitator pendidikan untuk mengidentifikasi siswa yang memiliki pola yang sama dalam belajar. Siswa yang awalnya memiliki kesamaan pola dalam belajar dan bertahan dengan kesulitan mereka untuk menyelesaikannya tugas dapat belajar dari siswa yang berhasil keluar dari tersebut kesulitan. Oleh karena itu, siswa tidak perlu benar-benar berubah strategi belajar mereka, tetapi dapat didekati dengan menggunakan strategi siswa yang telah berhasil keluar dari kesulitan selama proses pembelajaran.

Pada era modern saat ini marak dikembangkan media pembelajaran digital sebagai alat bantu dalam menunjang proses pembelajaran \cite{Muhasim2017}. Monsakun merupakan   salah   satu   media   pembelajaran   digital   yang   menggunakan   Tablet-PC sebagai medianya   dan   merupakan   media   pembelajaran   berbasis aritmatika dengan konsep problem posing yang digunakan di Sekolah Dasar Jepang \cite{Hasanah2015}. Monsakun mampu merekam kegiatan pengguna yang terjadi didalamnya melalui log data. Hal ini membuat sebuah proses menilai dan mengajar dapat dilakukan dengan sendirinya atau otomatis \cite{SupiantoHayashiHirashima2016}. Tentu saja adanya Monsakun dapat menjadi salah satu cara bagi guru untuk mengetahui bagaimana performa dari siswanya. Log data tersebut dapat diolah untuk berbagai kepentingan, seperti mengevaluasi proses belajar dari pengguna media pembelajaran itu sendiri. Evaluasi terhadap tingkah laku siswa saat menggunakan suatu media pembelajaran digital dapat membantu memetakan performa siswa dalam belajar. Melalui data tersebut, nantinya peranan dari media pembelajaran pun dapat terlihat. Pengolahan data terkait permasalahan tersebut dapat dilakukan dengan melakukan \textit{Clustering} atau pemetaan.

Sebelumnya, sebuah penelitian telah melakukan pengolahan pada log data yang dihasilkan oleh Monsakun untuk mengidentifikasi performa belajar siswa. Dalam penelitian ini, analisis dilakukan secara manual dengan mengambil kesimpulan melalui visualisasi dan persepsi peneliti. Siswa dipetakan menjadi empat kelompok berdasarkan pola siswa dalam menyelesaikan permasalahan dalam Monsakun. Hasil pemetaan dapat mengatakan bahwa siswa di setiap kelompok membutuhkan perlakuan yang berbeda-beda \cite{SupiantoHayashiHirashima2019}.  Penelitian  ini  membuktikan  bahwa  pemetaan penting dilakukan untuk mendapatkan umpan balik yang tepat sebagai dasar melakukan evaluasi. SOM merupakan algoritme yang cukup baik untuk melakukan proses \textit{Clustering} sekaligus memvisualisasikan hasilnya, SOM seringkali menimbulkan ambiguitas     mengenai     batasan-batasan     dari     Kelompok     yang dihasilkannya. Hal ini dibahas dalam penelitian yang dilakukan oleh \cite{Lee2019}, tentang \textit{Clustering} kualitas air tanah di Seoul, Korea Selatan. (SOM) tidak terikat oleh linearitas dari data (nonlinear). SOM merupakan salah satu algoritme yang paling sering digunakan untuk melakukan visualisasi yang merupakan salah satu cara analisis yang paling efisien untuk membantu menemukan pola maupun informasi dari data \cite{Kreuseler2002}. SOM melakukan pengurangan dimensi dari dimensi tinggi ke dimensi rendah dengan tetap mempertahankan karakteristik topologi. 

Sebelumnya penelitian untuk melakukan \textit{Clustering} pelajar menggunakan metode \textit{Self‐Organizing Maps} yang dilakukan oleh \cite{Ahmad2018} dengan data dari UTM \textit{Moodle E-Learning}. \cite{Bara2018} juga telah melakukan kelompok pelajar berdasarkan aktivitas belajar mereka dengan menggunakan metode \textit{Self-Organizing Maps} menggunakan data dari Universitas Teknologi Malaysia Moodle LMS selama satu semester. Algoritme \textit{Self‐Organizing Maps} yang digunakan pada kedua penelitian tersebut merupakan algoritme SOM klasik yang memiliki struktur neuron yang tetap dan didefinisikan pada awal algoritme. Pada kasus dengan data set yang karakteristiknya tidak diketahui dengan baik akan sulit untuk menentukan struktur yang tepat untuk mendapatkan hasil yang diinginkan.

Oleh karena itu, pada penelitian ini dilakukan percobaan melakukan \textit{Clustering} menggunakan algoritme \textit{Self-Organizing Maps} dengan m-ary-Tree. Pada \textit{SOM-m-ary-Tree}, fungsi graf adalah untuk menganalisis hasil \textit{Clustering} siswa dan umpan balik pada guru secara kuantitatif. Metode ini diharapkan dapat melakukan :

\begin{enumerate}
	\item Proses Clustering cara belajar siswa berdasarkan cara berfikir siswa dari tugas tugas yang telah dikerjakan dalam Media Pembelajaran Digital MONSAKU. Menurut \cite{SupiantoHayashiHirashima2017} cara menilai kepahaman harus memenuhi 5 Kriteria : \textit{Calculation, Formula, Sentence Structure, Story Type} dan \textit{Object}. \textit{Clustering} dimulai dari siswa yang terjelek sampai yang terbaik.
	
	\item 2.	Analisis kemiripan cara berfikir siswa yang terjelek dan terbaik dari siswa siswa yang terdekat dengan keduanya. Hal ini sangat bermanfaat sebagai umpan balik bagi guru dalam memberikan perlakuan yang cocok bagi siswa yang terjelek tanpa merombak total cara belajarnya yaitu dengan memberi perlakuan kemiripan seperti siswa yang terdekat dengan siswa tersebut.
\end{enumerate}

\section{Rumusan Masalah}
Berdasarkan latar belakang masalah yang telah diuraikan sebelumnya, dirumuskan permasalahan sebagai berikut :
\begin{enumerate}
	\item Bagaimana analisis pengaruh nilai parameter-parameter SOM-m-AT seperti Sigma, jumlah epoch, dan lain-lain terhadap kualitas Kelompok yang terbentuk untuk \textit{Clustering} cara berfikir siswa?
	
	\item Bagaimana analisis perbandingan hasil evaluasi metode SOM-m-AT dan SOM untuk \textit{Clustering} cara berfikir siswa (evaluasi \textit{Quantization Error} (QE) dan \textit{Topographic Error} (TE))?
	
	\item Bagaimana analisis hasil dari penerapan algoritma SOM-m-AT pada \textit{Clustering} siswa berdasarkan data aktivitas belajar?
	
	\item Bagaimana analisis \textit{similarity} penerapan algoritma SOM-m-AT untuk umpan balik ke guru dan siswa yang berupa sistem penasehat pembelajaran guru?
	
	\item Bagaimana persepsi guru Sekolah Dasar di Indonesia tentang sistem penasehat pembelajaran berdasarkan kemiripan siswa
\end{enumerate}

\section{Tujuan Penelitian}
Tujuan dari penelitian yang dilakukan adalah sebagai berikut:

\begin{enumerate}
	\item Untuk menganalisis karakteristik dari kelompok-kelompok yang diperoleh melalui proses \textit{Clustering} guna mengelompokkan performa belajar siswa dalam penggunaan media pembelajaran digital Monsakun.
	
	\item Untuk menganalisis pengaruh nilai \textit{Spread Factor}, \textit{Sigma}, dan jumlah \textit{epoch}
	terhadap kualitas kelompok yang terbentuk untuk \textit{Clustering} cara berfikir siswa.
	
	\item Menganalisis perbandingan hasil evaluasi metode SOM dan SOM-m-AT untuk \textit{Clustering} cara berfikir siswa yang menggunakan Monsakun.
	
	\item Menganalisis hasil analisis dari penggunaan kombinasi SOM-m-AT untuk \textit{Clustering} perfoma belajar siswa yang menggunakan Monsakun.
	
	\item Analisis hasil dari pengembangan algoritma SOM-m-AT untuk umpan balik ke guru  dan siswa yang berupa sistem penasehat pembelajaran guru di Indonesia?
\end{enumerate}

\section{Batasan Masalah}
Batasan dari penelitian ini adalah sebagai berikut:

\begin{enumerate}
	\item Data yang digunakan berupa aktivitas siswa dalam mengerjakan soal yang berasal dari pembelajaran interaktif Monsakun pada tahun 2014.
	
	\item Dataset yang digunakan adalah log data Monsakun siswa sekolah dasar di Jepang. 
\end{enumerate}