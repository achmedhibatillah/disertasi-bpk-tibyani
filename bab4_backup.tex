\chapter{METODE PENELITIAN}

\section{Tahapan Penelitian}
Secara garis besar, tahapan untuk melakukan penelitian ini dibagi menjadi:
\begin{enumerate}
    \item \textbf{Tahap persiapan}: Identifikasi dan perumusan masalah.
    \item \textbf{Tahap pengumpulan dan pengolahan data}.
    \item \textbf{Tahap analisis dan pembahasan}.
    \item \textbf{Kesimpulan}.
\end{enumerate}

Tahapan-tahapan yang dilakukan pada penelitian ini digambarkan pada Gambar~\ref{fig:tahapan_penelitian}.

\begin{enumerate}
    \item \textbf{Tahap Identifikasi dan Perumusan Masalah} \\
    Pada tahap ini, konsep proposal yang telah disusun direview kembali, terutama yang terkait dengan kajian literatur dan metodologi. Hal ini dilakukan untuk memastikan bahwa hasil kajian teori yang digunakan cukup kuat sebagai dasar argumentasi atau pijakan dalam penelitian ini. Selain itu, juga dimaksudkan agar metodologi yang akan dipakai dapat diimplementasikan dalam penelitian ini.

    Tahapan identifikasi masalah dilakukan untuk mengidentifikasi permasalahan yang akan diteliti. Melalui studi literatur ditemukan bahwa perlunya dilakukan penyesuaian komponen pembelajaran ataupun metode pembelajaran bagi pelajar berdasarkan perilaku dan tingkat pengetahuan mereka \citep{Ahmad2015}. 

    Pada tahap ini dilakukan studi kepustakaan untuk memperoleh informasi serta referensi yang digunakan sebagai pedoman dalam penelitian ini. Studi pustaka diperoleh dari berbagai artikel ilmiah seperti buku, jurnal, dan sumber daring yang berhubungan dengan topik penelitian ini yaitu SOM-m-aT. 

    Dalam proses pencarian literatur yang dijadikan sebagai pedoman, tahapan yang dilakukan meliputi: filtering judul dan abstrak, identifikasi bidang penelitian, metode pengelompokan, dan metode evaluasi pada literatur. Artikel ilmiah diperoleh melalui basis data seperti ERIC, Google Scholar, IEEE, JEDM, Portal Garuda, ScienceDirect, SpringerLink, dan ACM DL. Artikel hasil pencarian kemudian diseleksi lebih lanjut berdasarkan kesesuaian judul, bidang penelitian, kejelasan metode pengelompokan, serta kejelasan metode evaluasi yang digunakan.

    \item \textbf{Tahap Pengumpulan dan Pengolahan Data} \\
    Data yang digunakan pada penelitian ini berupa data log aktivitas yang dikumpulkan dari aktivitas belajar siswa sekolah dasar di Jepang berusia 6 tahun pada sistem media pembelajaran \textit{MONSAKUN}. \textit{MONSAKUN} dirancang sebagai media pembelajaran interaktif untuk \textit{problem-posing} berdasarkan model \textit{triplet structure} \citep{Hirashima2014}. 

    Sistem ini merekam aktivitas pelajar ketika menyelesaikan masalah, termasuk identitas pelajar, waktu, langkah, operator, slot, nomor kartu, susunan kartu, penilaian, dan error. Rincian variabel tersebut disajikan pada Tabel~\ref{tab:data_monsakun}.
\end{enumerate}


