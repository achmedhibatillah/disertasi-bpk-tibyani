\chapter{KESIMPULAN DAN SARAN}

Kami telah melakukan eksperimen awal dalam memvisualisasikan karakteristik siswa. Meskipun peta yang dihasilkan intuitif dan dalam beberapa hal informatif untuk memahami karakteristik siswa, peta tersebut hanya memberikan gambaran sekilas untuk satu tes tertentu. Dalam penelitian kami selanjutnya, kami berencana mengintegrasikan informasi dari peta ini ke dalam grafik kesamaan yang lebih baik menggambarkan kesamaan siswa secara hierarkis. Kami kemudian akan menggunakan grafik-grafik ini untuk secara otomatis menghasilkan saran pembelajaran yang dapat disesuaikan secara manual oleh guru dengan tujuan utama membantu siswa.
SOM-m-AT diusulkan sebagai metode baru untuk mempelajari kesamaan siswa dalam sistem bimbingan. Melalui beberapa eksperimen empiris, dapat diamati bahwa metode SOM-m-AT yang diusulkan outperforms SOM dalam efisiensi pembelajaran dan tingkat keberhasilan. Dalam penelitian ini, dengan metode SOM m-AT yang diusulkan, kami menemukan siswa dengan performa terburuk. Kita dapat mengetahui masalah yang dihadapi siswa, sehingga: menemukan siswa terdekat untuk mencoba menyelesaikan masalah yang terkait dengan strategi berpikir. Penelitian ini dapat memberikan informasi kepada guru untuk memberikan umpan balik yang sesuai berdasarkan masalah yang dihadapi siswa.


Berdasarkan hasil analisis data dan pembahasan, penelitian ini menyimpulkan bahwa guru sekolah dasar di Kota Malang secara umum memiliki persepsi positif terhadap sistem penasihat pembelajaran digital dan terbuka terhadap implementasinya dalam kegiatan pembelajaran di kelas. Hal ini didukung oleh analisis deskriptif, yang menunjukkan nilai rata-rata yang relatif tinggi pada variabel utama TAM, seperti Kegunaan yang Dirasakan (rata-rata = $15,30$) dan Kepercayaan Diri (rata-rata = $12,60$), menunjukkan bahwa guru menganggap sistem ini bermanfaat dan merasa percaya diri dalam menggunakannya. Berdasarkan hasil uji t parsial (Tabel 12), dari lima variabel independen, hanya Kompleksitas yang memiliki pengaruh signifikan terhadap Penggunaan Teknologi Aktual ($p = 0,037$) dengan koefisien negatif ($B = –0,367$), artinya semakin kompleks sistem tersebut dipersepsikan oleh guru, semakin kecil kemungkinan mereka untuk menggunakannya. Variabel lain seperti Kemudahan Penggunaan yang Dirasakan ($p = 0,442$), Kegunaan yang Dirasakan ($p = 0,133$), Keyakinan Diri ($p = 0,489$), dan Niat Perilaku untuk Menggunakan ($p = 0,081$) tidak menunjukkan efek yang signifikan secara statistik, meskipun niat menunjukkan kecenderungan mendekati signifikansi. Selain itu, hasil uji F (Tabel 13) menunjukkan bahwa model regresi secara keseluruhan secara statistik signifikan, dengan nilai F sebesar 6.635 dan tingkat signifikansi $p = 0.002$. Ini berarti semua variabel independen secara kolektif memiliki pengaruh signifikan terhadap Penggunaan Teknologi Aktual. Hasil ini memperkuat validitas model regresi dan menunjukkan bahwa variabel yang diukur berkontribusi dalam menjelaskan perilaku guru dalam mengadopsi teknologi. Kesimpulannya, meskipun tidak semua variabel secara individu menunjukkan pengaruh yang signifikan, kombinasi persepsi guru tentang kemudahan penggunaan, kegunaan, dan kepercayaan diri memainkan peran penting dalam mendorong penggunaan sistem yang sebenarnya. Temuan ini menyoroti pentingnya mengurangi kompleksitas sistem dan menyediakan pelatihan yang memadai untuk mendukung adopsi teknologi yang lebih luas. Studi ini memberikan landasan yang kuat untuk mengembangkan kebijakan pendidikan berbasis teknologi yang selaras dengan penerimaan guru dan realitas pembelajaran di kelas sekolah dasar Indonesia.
